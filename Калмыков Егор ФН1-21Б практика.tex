\documentclass[12pt]{article}
\usepackage[utf8]{inputenc}
\usepackage[russian]{babel}
\usepackage{amsmath,amssymb}
\usepackage{graphics}
\usepackage{pbox}
\usepackage[x11names]{xcolor}
\definecolor{brightmaroon}{rgb}{0.76, 0.13, 0.28}
\definecolor{royalazure}{rgb}{0.0, 0.22, 0.66}
\usepackage[colorlinks=true,linkcolor=royalazure]{hyperref}
\usepackage{tikz, tkz-fct, pgfplots}
\usetikzlibrary{arrows}
\usepackage{geometry}
\geometry{
	a4paper,
	total={170mm,257mm},
	left=20mm,
	top=20mm
} 
\usepackage[labelsep=period]{caption}

% ---------------------------------------------------------------------------
\newcommand{\eps}{\varepsilon}
\newcommand\tline[2]{$\underset{\text{#1}}{\text{\underline{\hspace{#2}}}}$}

% --------------------------------------------------------------------------- 
\graphicspath{{img/}}
\begin{document}
\pagestyle{empty}
\centerline{\large Министерство науки и высшего образования}	
\centerline{\large Федеральное государственное бюджетное образовательное}
\centerline{\large учреждение высшего образования}
\centerline{\large ``Московский государственный технический университет}
\centerline{\large имени Н.Э. Баумана}
\centerline{\large (национальный исследовательский университет)''}
\centerline{\large (МГТУ им. Н.Э. Баумана)}
\hrule
\vspace{0.5cm}
\begin{figure}[h]
\center
\includegraphics[height=0.35\linewidth]{bmstu-logo-small.png}
\end{figure}
\begin{center}
	\large	
	\begin{tabular}{c}
		Факультет ``Фундаментальные науки'' \\
		Кафедра ``Высшая математика''		
	\end{tabular}
\end{center}
\vspace{0.5cm}
\begin{center}
	\LARGE \bf	
	\begin{tabular}{c}
		\textsc{Отчёт} \\
		по учебной практике \\
		за 1 семестр 2021---2022 гг.
	\end{tabular}
\end{center}
\vspace{0.5cm}
\begin{center}
	\large
	\begin{tabular}{p{5.3cm}ll}
		\pbox{5.45cm}{
			Руководитель практики,\\
			ст. преп. кафедры ФН1} 	& \tline{\it(подпись)}{5cm} & Кравченко О.В. \\[0.5cm]
		студент группы ФН1--11 		& \tline{\it(подпись)}{5cm} & Калмыков Е.А.
	\end{tabular}
\end{center}
\vfill
\begin{center}
	\large	
	\begin{tabular}{c}
		Москва, 
		2022 г.
	\end{tabular}
\end{center}

\newpage	
\tableofcontents

\newpage
\section{Цели и задачи практики}	
\subsection{Цели}
--- развитие компетенций, способствующих успешному освоению материала бакалавриата и необходимых в будущей профессиональной деятельности.

\subsection{Задачи}
\begin{enumerate}
\item Знакомство с программными средствами, необходимыми в будущей профессиональной деятельности.
\item Развитие умения поиска необходимой информации в специальной литературе и других источниках.
\item Развитие навыков составления отчётов и презентации результатов.
\end{enumerate}

\subsection{Индивидуальное задание}	
\begin{enumerate}
\item Изучить способы отображения математической информации в системе вёртски \LaTeX.
\item Изучить возможности  системы контроля версий \textsf{Git}.
\item Научиться верстать математические тексты, содержащие формулы и графики в системе \LaTeX.
Для этого, выполнить установку свободно распространяемого дистрибутива \textsf{TeXLive} и оболочки \textsf{TeXStudio}.
\item Оформить в системе \LaTeX типовые расчёты по курсе математического анализа согласно своему варианту.
\item Создать аккаунт на онлайн ресурсе \textsf{GitHub} и загрузить исходные \textsf{tex}--файлы 
и результат компиляции в формате \textsf{pdf}.
\end{enumerate} 

\newpage
\section{Отчёт}
Актуальность темы продиктована необходимостью владеть системой вёрстки \LaTeX и средой вёрстки \textsf{TeXStudio} для
отображения текста, формул и графиков. Полученные в ходе практики навыки могут быть применены при написании
курсовых проектов и дипломной работы, а также в дальнейшей профессиональной деятельности.

Ситема вёрстки \LaTeX содержит большое количество инструментов (пакетов), упрощающих отображение информации в различных 
сферах инженерной и научной деятельности. 

\newpage
\section{Индивидуальное задание}
\subsection{Пределы и непрерывность.}
% ---------------------------- 1 ----------------------------------
\subsubsection*{\center Задача № 1.}
{\bf Условие.~}
Дана последовательность $a_{n}=\dfrac{2n+1}{3n-5}$ и число $c=\dfrac{2}{3}$.
Доказать, что $$\lim\limits_{n\rightarrow\infty}a_n=c,$$
а именно, для каждого $\eps>0$ найти наименьшее натуральное число $N{=}N(\eps)$ такое, что $|a_{n}-c|<\eps$ для всех $n>N(\eps)$.
Заполнить таблицу:
\begin{center}
\begin{tabular}{ | p{25pt} | c | c | c | c |}
\hline
$\varepsilon$& $0{,}1$ & $0{,}01$ & $0{,}001$ \\
\hline
$N(\varepsilon)$ &   &   &\\
\hline
\end{tabular}    
\end{center}
{\bf Решение.~}
Рассмотрим неравенство $a_n-c<\eps,\,\forall\eps>0$, учитывая выражение для $a_n$ и значение $c$ из условия варианта,
получим:
$$
\biggl|\dfrac{2n+1}{3n-5}-\frac{2}{3}\biggr| < \eps ;
$$
$$
\biggl|\frac{6n+3-6n+10}{9n-15}\biggr| < \eps ;
$$
$$
\biggl|\frac{13}{9n-15}\biggr| < \eps .
$$
$$
\begin{array}{c}
\dfrac{13}{9n-15} < \eps ;             \\[8pt]
n > \dfrac{13}{9\eps} + \dfrac{15}{9} . \\[8pt]
\end{array}
$$
\center Заполним таблицу:
\begin{center}
\begin{tabular}{ | p{25pt} | c | c | c | c |}
\hline
$\varepsilon$& $0{,}1$ & $0{,}01$ & $0{,}001$ \\
\hline
$N(\varepsilon)$& $16$ & $146$ & $1446$\\
\hline
\end{tabular}
\end{center} 
% ---------------------------- 2 ----------------------------------
\subsubsection*{\center Задача № 2.}
{\bf Условие.~}
Вычислить пределы функций
$$
\begin{array}{cc}
\text{\bf(а):} & \lim\limits_{x\rightarrow-1}\dfrac{x^3+5x^2+7x+3}{x^3+4x^2+5x+2}; \\[10pt]
\text{\bf(б):} & \lim\limits_{x\rightarrow+\infty}\dfrac{2x^2\sqrt{\,1+9x^4}}{(\sqrt{\,x}+1)^2(\sqrt[3]{\,x}-2)^3}; \\[10pt]
\text{\bf(в):} & \lim\limits_{x\rightarrow8}\dfrac{x-8}{\sqrt[3]{\,x}-2}; \\[10pt]
\text{\bf(г):} & \lim\limits_{x\rightarrow+0}\biggl(2-5^{\arcsin{x^2}}\biggr)^{\frac{1}{\sin{x}\cdot x}}; \\[10pt]
\text{\bf(д):} & \lim\limits_{x\rightarrow+0}\biggl(\dfrac{\sin{3x}}{\sin{2x}}\biggr)^{(\cos{x})^2}; \\[10pt]
\text{\bf(е):} & \lim\limits_{x\rightarrow1}\dfrac{\ln{(3-2x)}}{\arctan{(3x-3)}}.
\end{array}
$$
% ---------------------------- 2а --------------------------------
{\bf Решение:~}
\text{\bf(а):}
$$
\begin{array}{l}
\lim\limits_{x\rightarrow-1}\dfrac{x^3+5x^2+7x+3}{x^3+4x^2+5x+2} = 
\lim\limits_{x\rightarrow-1}\dfrac{(x+1)(x^2+4x+3)}{(x+1)(x^2+3x+2)} = 
\lim\limits_{x\rightarrow-1}\dfrac{(x+1)(x+3)}{(x+1)(x+2)} = 2.
\end{array}
$$
% ---------------------------- 2б --------------------------------
\text{\bf(б):}
$$
\begin{array}{l}
\lim\limits_{x\rightarrow+\infty}\dfrac{2x^2\sqrt{\,1+9x^4}}{(\sqrt{\,x}+1)^2(\sqrt[3]{\,x}-2)^3} = 
\lim\limits_{x\rightarrow+\infty}\dfrac{(2x^2\sqrt{\,1+9x^4})(2x^2\sqrt{\,1+9x^4})}
{(\sqrt{\,x}+1)^2(\sqrt[3]{\,x}-2)^3(2x^2\sqrt{\,1+9x^4})} =
\end{array}
$$ \\
$$
\begin{array}{l}
= \lim\limits_{x\rightarrow+\infty}\dfrac{-x^4(5+\frac{1}{x^4})}
{x(1+\frac{1}{\sqrt{\,x}})^2\cdot x(1-\frac{2}{\sqrt[3]{\,x}})^3\cdot x^2(2-\sqrt{\,\frac{1}{x^4}+9})} = 5.
\end{array}
$$
% ---------------------------- 2в --------------------------------
\text{\bf(в):}
$$
\begin{array}{l}
\lim\limits_{x\rightarrow8}\dfrac{x-8}{\sqrt[3]{\,x}-2} = \lim\limits_{x\rightarrow8}\dfrac{\sqrt[3]{\,x^3}-2^3}{\sqrt[3]{\,x}-2} = 
\lim\limits_{x\rightarrow8}\biggl(\sqrt[3]{\,x^2}+2\sqrt[3]{\,x}+4\biggl) = 12.
\end{array}
$$
% ---------------------------- 2г --------------------------------
\text{\bf(г):}	
$$
\begin{array}{l}
\lim\limits_{x\rightarrow+0}\biggl(2-5^{\arcsin{x^2}}\biggr)^{\frac{1}{\sin{x}\cdot x}} = 
\lim\limits_{x\rightarrow+0}\biggl(2-5^{x^2}\biggr)^{\frac{1}{x^2}} =
\lim\limits_{x\rightarrow+0}\biggl(1+(1-5^{x^2})\biggr)^{\frac{1-5^{x^2}}{x^2(1-5^{x^2})}} = 
\end{array}
$$ \\
$$
\begin{array}{l}
= e^{\biggl(\lim\limits_{x\rightarrow+0}\frac{1-5^{x^2}}{x^2}\biggl)} = e^{\ln{5}} = \dfrac{1}{5}.
\end{array}
$$
% ---------------------------- 2д --------------------------------
\text{\bf(д):}
$$
\begin{array}{l}
\lim\limits_{x\rightarrow+0}\biggl(\dfrac{\sin{3x}}{\sin{2x}}\biggr)^{(\cos{x})^2} = 
\lim\limits_{x\rightarrow+0}\biggl(\dfrac{3x}{2x}\biggr) = \dfrac{3}{2}.
\end{array}
$$
% ---------------------------- 2е --------------------------------
\text{\bf(е):}
$$
\begin{array}{l}
\lim\limits_{x\rightarrow1}\dfrac{\ln{(3-2x)}}{\arctan{(3x-3)}} = 
\biggl|
\begin{array}{l}
t=x-1 \\ t\rightarrow0
\end{array}
\biggr|
= \lim\limits_{t\rightarrow0}\dfrac{\ln{(1-2t)}}{\arctan{(3t)}} = \lim\limits_{t\rightarrow0}\dfrac{-2t}{3t} = -\dfrac{2}{3}.
\end{array}
$$
\newpage
% ---------------------------- 3----------------------------------
\subsubsection*{\center Задача № 3.}
{\bf Условие.~}\\
\text{\bf(а):} Показать, что данные функции
$f(x)$ и $g(x)$ являются бесконечно малыми или бесконечно большими
при указанном стремлении аргумента. \\
\text{\bf(б):} Для каждой функции $f(x)$ и $g(x)$ записать главную часть
(эквивалентную ей функцию)  вида $C(x-x_0)^{\alpha}$ при $x\rightarrow x_0$ или $Cx^{\alpha}$
при $x\rightarrow\infty$, указать их порядки малости (роста). \\
\text{\bf(в):} Сравнить функции $f(x)$ и $g(x)$ при указанном стремлении.
\begin{center}
	\begin{tabular}{|c|c|c|}
		\hline
		№ варианта & функции $f(x)$ и $g(x)$ & стремление \\[6pt]
		%\hline
		12 & $f(x) = \sqrt{\,x+\frac{1}{x}}-\sqrt{\,x}, ~g(x)=\dfrac{\arctan(1-x)\sin{\frac{1}{x}}}{x}$ & $x\rightarrow+\infty$ \\
		\hline
	\end{tabular}
\end{center}
{\bf Решение.~}\\
~Выделим главные части функций $f(x)$ и $g(x)$:
$$
\begin{array}{cc}
f(x) = \sqrt{\,x+\frac{1}{x}}-\sqrt{\,x} = \sqrt{\,x}(\sqrt{\,1+\frac{1}{x^2}}-1) \thicksim 
\sqrt{\,x}\cdot \dfrac{1}{2x^2} \thicksim \dfrac{1}{2}\cdot x^{-\frac{3}{2}}.
\end{array}
$$
~Тогда при $x\rightarrow+\infty:\ \ c=\dfrac{1}{2};\ \ \alpha={-\frac{3}{2}}.$
$$
\begin{array}{cc}
g(x) = \dfrac{\arctan(1-x)\sin{\frac{1}{x}}}{x} \thicksim -\dfrac{\pi}{2}\cdot x^{-2}.
\end{array}
$$
~Тогда при $x\rightarrow+\infty:\ \ c=-\dfrac{\pi}{2};\ \ \alpha=-2.$
$$
$$
~Покажем, что $f(x)$ и $g(x)$ бесконечно малые функции:
$$ 
\lim\limits_{x\rightarrow+\infty}f(x) =
\lim\limits_{x\rightarrow+\infty}\dfrac{1}{2}\cdot x^{\frac{3}{2}} = \dfrac{1}{2}\lim\limits_{x\rightarrow+\infty}x^{-\frac{3}{2}} = 0.
$$
$$
\lim\limits_{x\rightarrow+\infty}g(x) = -\lim\limits_{x\rightarrow+\infty}\dfrac{\pi}{2}\cdot x^{-2} =
-\dfrac{\pi}{2}\cdot\lim\limits_{x\rightarrow+\infty}x^{-2} = 0.
$$ \\
~$k_f = -\dfrac{3}{2}$ - порядок малости БМФ $f(x)$ относительно $x\rightarrow+\infty$ . \\
$k_g = -2$ - порядок малости БМФ $g(x)$ относительно $x\rightarrow+\infty$ .
$$
$$
~Для сравнения функций $f(x)$ и $g(x)$ рассмотрим предел их отношения:
$$
\lim\limits_{x\rightarrow+\infty}\dfrac{f(x)}{g(x)}.
$$
Применим эквивалентности, получим:
$$
\lim\limits_{x\rightarrow+\infty}\dfrac{f(x)}{g(x)} = 
\lim\limits_{x\rightarrow+\infty}\dfrac{{\sqrt{\,x}}}{-\pi} = - \infty .
$$
Отсюда следует, что $f(x) = o(g(x))$.
%=========================================================================
\newpage
\addcontentsline{toc}{section}{Список литературы}
\begin{thebibliography}{99}
\bibitem{book01} Львовский С.М. Набор и вёрстка в системе \LaTeX, 2003 c.
\bibitem{book02} Котельников И.А., Чеботаев П.З. \LaTeX~по-русски.
\bibitem{book03} Чебарыков М.С Основы работы в системе \LaTeX.
\end{thebibliography}
\end{document}